\documentclass[10pt]{article}
% \usepackage{mathabx}
\usepackage{titling}
\usepackage{multicol}
\usepackage{amssymb}
\usepackage{amsmath}
\usepackage{physics}
\usepackage[bottom]{footmisc}

\usepackage{graphicx}
\usepackage[margin=0.5in]{geometry}
\usepackage{float}
\usepackage{listings}
\usepackage[utf8]{inputenc}
\usepackage[parfill]{parskip}  
\usepackage{siunitx}
\usepackage[dvipsnames]{xcolor}

\usepackage{cite}
\usepackage{caption}
\usepackage{tabularx}
\usepackage{booktabs}
\usepackage{hyperref}
\usepackage{cleveref}

\captionsetup{width=0.6\linewidth}
\renewcommand{\thefootnote}{\fnsymbol{footnote}}


\newcommand{\rhomax}{\rho_{\text{max}}}
\newcommand{\relerr}{\epsilon_{\text{rel}}}
\newcommand{\bigO}[1]{\mathcal{O}(#1)}
\newcommand{\rmd}{{\mathrm d}}

\newcommand{\oder}[2]{\frac{{\rm d}#1}{{\rm d}#2}}
\newcommand{\odder}[2]{\frac{{\rm d}#1}{{\rm d}#2}}
\newcommand{\pd}[1]{\frac{\partial}{\partial #1}}
\newcommand{\pdt}[2]{\frac{\partial #1}{\partial #2}}
\newcommand{\pddt}[2]{\frac{\partial^2 #1}{\partial #2^2}}
\newcommand{\tpddt}[2]{\tfrac{\partial^2 #1}{\partial #2^2}}

\newcommand{\npdt}[3]{\frac{\partial^{#3} #1}{\partial #2^{#3}}}

\newcommand{\cop}{a^{\dagger}}
\newcommand{\aop}{a}
\usepackage{simplewick}

\graphicspath{{../figures/}}

\newcommand\myshade{50}
\colorlet{mylinkcolor}{violet}
\colorlet{mycitecolor}{YellowOrange}
\colorlet{myurlcolor}{Aquamarine}

\usepackage{eso-pic}

\begin{document}
\title{FYS4480 Quantum mechanics for many-particle systems \\
--- Project 1 ---}

\author{Erlend Lima, Lasse Lorentz Braseth and Halvard Sutterud}
\date{October 2018}
\maketitle{\begin{center}\end{center}}
\pagenumbering{gobble}

% \AddToShipoutPictureBG*{\includegraphics[width=\paperwidth]{test.pdf}};
% \clearpage

\thispagestyle{empty}

\begin{figure}[htpb]
    \centering
    \includegraphics[width=0.9\linewidth]{../figures/UiO_Segl_300dpi.png}
    \label{fig:name}
\end{figure}


\newpage

\begin{abstract}
    Hei
\end{abstract}
    
\tableofcontents
% \newpage
\begin{multicols}{2}

\pagenumbering{arabic}
%%%%%%%%%%%%%%%%%%%%%%%%%%%%%%%%%%%%%%%%%%%%%%%%%%%%%%%%%%
\section{Introduction}
\label{sec:introduction}

%%%%%%%%%%%%%%%%%%%%%%%%%%%%%%%%%%%%%%%%%%%%%%%%%%%%%%%%%%
\section{Theory}
\label{sec:theory}

\subsection{Second quantization}%
\label{sub:second_quantization}

We introduce the second quantization formalism to simplify the notation and
calculations. Second quantization introduces creation and annihilation
operators, which we will represent as $a^{\dagger}_p$ and $a_p$
respectively. They work on a slater determinant in the following way, 

\begin{align*}
    \cop_p \ket{\alpha_1\alpha_2\dots\alpha_n} &=
    \ket{p\alpha_1\alpha_2\dots\alpha_n}\\
    \aop_p \ket{p\alpha_1\alpha_2\dots\alpha_n} &=
    \ket{\alpha_1\alpha_2\dots\alpha_n}.
\end{align*}

The $\cop_p$ thus has the effect of creating a particle in the state $p$, and
$\aop_p$ to remove a particle from the same state. In addition, we have 

\begin{align*}
    \cop_p \ket{p\alpha_1\alpha_2\dots\alpha_n} &= 0\\
    \aop_p \ket{\alpha_1\alpha_2\dots\alpha_n} &= 0,
\end{align*}

i.e. the creation operator cannot create a particle in an occupied state
(the Pauli Exclusion principle), and in the second case the annihilation
operator cannot remove a particle not present (common sense).



\subsection{Wicks theorem}%
\label{sub:wicks_theorem}

Wicks theorem states that a product of $N$ operators can be expressed as the
normal-ordered product of the operators plus the normal-ordered products of
every possible permutation of one to $N/2$ contractions of the same
operators.

only
need to consider the sum over normal products of all contractions between
two and two operators. 

\subsubsection{Generalized wicks theorem}

When using the particle-hole formalism, we need to give a small
reformulation of wicks theorem. Main changes: We consider 




%%%%%%%%%%%%%%%%%%%%%%%%%%%%%%%%%%%%%%%%%%%%%%%%%%%%%%%%%%
\subsection{Part a (to be renamed)}

% a)
Our basis in the second
quantization are slater determinants consisting of s-wave ($l=0$) hydrogen like single particle
functions $R_{n0}(r)$, given by

\begin{equation}
    R_{n0}(r) = \left(\frac{2Z}{n}\right)^{3/2}
    \sqrt{\frac{(n-1)!}{2n(n!)}}L_{n-1}^1 (\frac{2Zr}{n})
    \exp(-\frac{Zr}{n}),
\end{equation}

where $L_{n-1}^1(r)$ are the so-called Laguerre polynomials. We then have a
degeneracy for the states with quantum number $n$ of $2$, since we only
consider s-waves. 


We will be working with the helium atom. We define our single-particle
Hilbert space to consist of the single-particle orbits $1s$, $2s$, and $3s$
with corresponding spin degeneracies, and consider that we will be working
with a two-electron system.  We will number the six
possible single particle states $\ket{nm_s}$ as seen in
\cref{tab:he_numbering}. The Fermi level is then defined to be $F = 2$. Our
ansatz for the ground state $\ket{c} = \ket{\Phi_0}$ of the helium atom is
then two electrons placed in the ground state with $n=1$, with opposite
spin values, which we can then represent as
%
\begin{equation}
    \ket{c} = \prod_{i=1}^{F} \cop_i\ket{0} = \cop_2\cop_1\ket{0} 
\end{equation}


% TODO: Skrive notasjon for indeksering over og under ferminivået.
Assuming that the total spin is projection of the possible states is $M_S =
0$, the possible one-particle-one-hole excitations are $\ket{\Phi_i^a} &=
\cop_a\aop_i\ket{c}$ in these combinations,
%
\begin{align*}
                   \ket{\Phi_1^3} &= \cop_3\aop_1\ket{c}\qc
                   &\ket{\Phi_1^5} &= \cop_5\aop_1\ket{c}\\
                   \ket{\Phi_2^4} &= \cop_4\aop_2\ket{c}\qc
                   &\ket{\Phi_2^6} &= \cop_6\aop_2\ket{c},
\end{align*}
%
while the possible two-particle-two-holes excitations are $\ket{\Phi_{ij}^{ab}} &=
\cop_b\cop_a\aop_j\aop_i\ket{c}$ in the following combinations,
%
\begin{align*}
    \ket{\Phi_{12}^{34}} &= \cop_4\cop_3\aop_2\aop_1\ket{c}\qc
    & \ket{\Phi_{12}^{56}} &= \cop_6\cop_5\aop_2\aop_1\ket{c}\\
    \ket{\Phi_{12}^{36}} &= \cop_6\cop_3\aop_2\aop_1\ket{c}\qc
    & \ket{\Phi_{12}^{45}} &= \cop_5\cop_4\aop_2\aop_1\ket{c}.
\end{align*}

% b)


\begin{table}[H]
    \centering
    \caption{Here $\uparrow$ corresponds to magnetic spin quantum number $m_s =
    1/2$, while $\downarrow$ corresponds to $m_s = -1/2$. }
    \label{tab:he_numbering}
    \begin{tabular}{c | c c c c c c}
        \toprule
        \midrule
        index & 1 & 2 & 3 & 4 & 5 & 6 \\
        state & $\ket{1\uparrow}$ &
         $\ket{1\downarrow}$ &
         $\ket{2\uparrow}$   &
         $\ket{2\downarrow}$ &
         $\ket{3\uparrow}$   &
         $\ket{3\downarrow}$ &
         \bottomrule
    \end{tabular}
\end{table}


\subsection{Part b (to be renamed)}

The Hamiltonian in a second-quantized form is given as

\begin{equation}
    \hat H = \sum_{pq} \mel{p}{\hat h}{q} \cop_p\aop_q +
    \frac{1}{4}\sum{pqrs}\mel{pq}{\hat v}{rs} \cop_p\cop_q\aop_s\aop_r,
\end{equation}

where the sums run both above and below the fermi surface. Splitting these
sums give us the  BLA BLA


Or maybe we should split it into  non-interacting and interacting?
\begin{equation}
    \hat H_0 = 
    \sum_{pq > F}\mel{p}{h_0}{q} 
\end{equation}

\section{Methods and implementation}
\label{sec:methods}
%%%%%%%%%%%%%%%%%%%%%%%%%%%%%%%%%%%%%%%%%%%%%%%%%%%%%%%%%%
\section{Results}
\label{sec:results}

%%%%%%%%%%%%%%%%%%%%%%%%%%%%%%%%%%%%%%%%%%%%%%%%%%%%%%%%%%
\section{Discussion}
\label{sec:discussion}


%%%%%%%%%%%%%%%%%%%%%%%%%%%%%%%%%%%%%%%%%%%%%%%%%%%%%%%%%%
\section{Conclusions}


\bibliographystyle{plain}
% \bibliography{../FYSSTK4155}{}
\bibliography{FYS4480.bib}

\end{multicols}
\end{document}j
